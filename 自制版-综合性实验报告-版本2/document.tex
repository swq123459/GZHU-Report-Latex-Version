\documentclass[10pt,a4paper]{article}
\usepackage{graphicx}
\usepackage{ctex}
\usepackage{indentfirst}
%\graphicspath{{chapter/}{figures/}}
\usepackage{CJK}

%\usepackage[colorlinks,linkcolor=red]{hyperref}%超链接

\usepackage{fancyhdr}  %使用fancyhdr包自定义页眉页脚
%\pagestyle{empty}
\pagestyle{fancy}
%\pagestyle{plain}%没有页眉,页脚放页数
\renewcommand{\headrulewidth}{0.5pt}
\renewcommand{\footrulewidth}{0.4pt}
\lhead{}
\chead{}
\rhead{}
\lfoot{}
\cfoot{\thepage}
\rfoot{}
\usepackage{booktabs}%表格用

\usepackage{float}%可以用于禁止浮动体浮动

%目录超链接
\usepackage[colorlinks,linkcolor=black,anchorcolor=blue,citecolor=black]{hyperref}

\usepackage{listings}%可以插入代码
\usepackage{xcolor}%语法高亮支持
%代码格式
\definecolor{dkgreen}{rgb}{0,0.6,0}
\definecolor{gray}{rgb}{0.5,0.5,0.5}
\definecolor{mauve}{rgb}{0.58,0,0.82}
\lstset{ %
	%	language=Python,                % the language of the code
	breaklines,%自动折行
	%extendedchars=false%解决代码跨页时,章节标题,页眉等汉字不显示的问题
	keepspaces=false,  
	%tabsize=4 %设置tab空格数
	showspaces=false,  %不显示空格
	showtabs=false,  
	showstringspaces=true, 
	numbers=left, 
	basicstyle=\footnotesize, 
	numberstyle=\tiny, 
	numbersep=5pt, 
	keywordstyle= \color{ blue!70},%关键字颜色
	commentstyle= \color{red!50!green!50!blue!50},%注释颜色 
	frame=shadowbox, % 边框格式:阴影效果
	rulesepcolor= \color{ red!20!green!20!blue!20} ,
	escapeinside=``, % 英文分号中可写入中文
	xleftmargin=2em,xrightmargin=2em, aboveskip=1em,%设置页边距
	framexleftmargin=2em
}




%设置页面格式
\usepackage[left=2.0cm, right=2.0cm, top=2.0cm, bottom=2.0cm]{geometry}
\begin{document}
%%%%%%%%%%%%%%%%%%%%%%%%%%%%%%
%% 封面部分
%%%%%%%%%%%%%%%%%%%%%%%%%%%%%%
\begin{titlepage}
	\centering
	\includegraphics[width=0.2\textwidth]{sf1.png}\par
	\vspace{0.5cm}
	\includegraphics[width=0.4\textwidth]{sf.png}\par
	\vspace{0.1cm}
	{\scshape\LARGE Guangzhou University \par}
	\vspace{1cm}
	{\kaishu\huge 基于监控视频的\par}
	\vspace{0.5cm}
	{\kaishu\huge 课堂自动签到系统的设计\par}
	\vspace{5cm}
	{\LARGE\bfseries 数字图像处理\par}
	\vspace{1cm}
	{\fangsong\Large\itshape 苏伟强\par}
	\vspace{1cm}
	{1507400051}\par
	\fangsong{电信151}\par
	指导教师	\textsc{胡晓}
	\vfill
% Bottom of the page
	{\large \today\par}
\end{titlepage}
\tableofcontents
\newpage


\begin{thebibliography}{4}
	\bibitem{Butterworth}Butterworth S. On the theory of filter amplifiers[J]. Wireless Engineer, 1930, 7(6): 536-541.
\end{thebibliography}


\end{document}
